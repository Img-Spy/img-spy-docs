\cleardoublepage
\phantomsection
\chapter*{Conclusions}

The development of a cross-platform digital forensics applications is a long
process with many key decisions. The first one is to choose a technological
environment. This task is not that easy due to the great amount of
very-different solutions. But, nowadays web technologies are standing strong in
terms of cross-platform compatibility. \textbf{A well-known solution} that many
companies are using today is \textbf{Electron} \cite{electron-web}, which uses
Chromium as library to instantiate Chrome windows. 

Once the work environment is well-defined, an overview of the application 
should be provided. The software architecture defines it by using black boxes
and defining their interactions. Model-view-controller (MVC) is a mature
widely-used architecture. Its main disadvantage is that becomes complex when
increasing the number of views with interaction between themselves. Since
this is the case of the application we have developed, MVC is not a good
architecture in this case.

The direct evolution of MVC is Flux, which was defined by Facebook to fix its
scalabilty problem. In order to solve this problem, Flux purposes the use of
a unique state to render whole the application. In this project we have
used \textbf{React-Redux} to build  \textbf{flux-like architecture}
Nevertheless, other libraries, such as Redux-Observables, have been used in
order to address React-Redux lacks.

With those good bases, a complete application can be build but there is still a
need to implement the computationally complex operations of the digital 
forensics work flow. Those have to be coded in a low programming language to
guarantee a good efficiency. The Sleuth Kit provides a C library that
implements those operations. Therefore, \textbf{The Sleuth Kit JavaScript, a
Node.js wrapper, has been created} to let JavaScript use those operations.

Finally, \textbf{Img-Spy, the digital forensics application developed to fulfill
the goal of this project, defines three tools}: \textit{Explorer},
\textit{Timeline} and \textit{Search}. Those tools let an investigator to
analyze the file system of an image in a non-intrusive way, create a timeline
based on actions performed on the disk and search which files contain a
specific string.

During the development process, many interesting or difficult tasks where found,
for instance, watching changes on the file system of user's computer, performing
actions in parallel using JavaScript and defining a user-friendly interface
supporting, for example, resize-panels or multiple themes.

Looking ahead, many optimizations and features can be added to Img-spy. First,
\textit{Explorer} tool can be improved in terms of performance adding Redux
selectors and reducing React rerenders. Also, the hash is always computed
without asking the user, consuming too many resources that are not always
required.

\textit{Timeline} tool can be also improved by adding graphs to represent those
actions in a more visual way. Add filters can also help investigators to remove
useless information. For instance, a useful filter could be to see just the
files of a specific search result.

In many cases, search files by name is also needed. Therefore, \textit{Search}
tool can implement this operation. More custom searches can be executed using
Regex.

There is a saying that “Programming can be fun, so can cryptography; however
they should not be combined” \cite{code-complete}. So regarding code quality,
comments can be added to help new developers understand easily how the software 
works.

Finally, new tools can be added in order to detect mismatches on file 
extensions or to help investigators to write the final report. Just take always
into account that, as Gordon Bell said:

\begin{quote}
	“Every big computing disaster has come from taking too many ideas and putting them in one place”.
\end{quote}
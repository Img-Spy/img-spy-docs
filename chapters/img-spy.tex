\chapter{Img-spy}
\label{S:img-spy}

Imgspy is GUI (graphic user interface) that performs digital forensic analysis
using the sleuthkit as library. It is coded using Typescript for the logic,
HTML5 to define the view and Sass the look and feel. Javascript and CSS are not 
directly used because Typescript increases the code reliability since types 
can be defined while Sass adds flexibility to the style definition, such as, functions and inheritance.

The source code of the project is within the \textit{src} folder. Typescript
and Sass need to be compiled because they cannot be directly interpreted. For
this reason, \textit{dist} folder contains all the source code changing the 
Typescript files for Javascript ones and Sass for CSS.

The application is divided in a frontend and a backend. The last one is 
responsible of the costly operations and the window management, while frontend
shows the user interface.

\begin{description}
	\item[main] Typescript files of the backend.
	\item[app] Typescript and HTML files of the frontend. They contain the view
	and the application logic.
	\item[lib] Javascript libraries not uploaded to npm.
	\item[i18n] Internationalization files.
	\item[style] Sass files. They contain the look and fell of the application.
\end{description}

%%% RESUMEN DE COMO FUNCIONA EL BACKEND

Electron starts a process called Main which loads a Node.js script. It is in
charge of manage the whole application. This process launches two instances, the
WindowManager and the FileSystemWorkersManager.

The WindowManager creates all windows shown to the user and handles the
communication between them. Each window run in another process Renderer and it
is an instance of a google chrome window that also has all the  node libraries
included. Windows are CaseSelector, Case and Settings.

To perform communications, windows can listen to channel (a string) and send
data using the object ipcRenderer from Electron. A react component called
WindowEvent was created to listen those channels and make the management easier.

Also, it has to maintain a dictionary of all the windows to close the Main
process when all of them are closed. To do so, each type of window has it's own
unique name. This fact also guarantees that no more than one window of each type
is shown on the same time. 

Regarding FileSystemWorkersManager, it is the one that performs the digital
forensic tasks. Since Main process must not be frozen performing long
calculations, many child processes are called waiting for those tasks. The
number depends on how many CPU cores have the computer running the program.

This instance also has a FIFO queue of queries that stores them if all the
workers are busy. Each query is encapsulated in a message containing an UUIDv1
to fix a possible order mismatch due to asynchronous responses.

% De todo esto podria hacer una figura para que quedara más claro

%%% RESUMEN DE COMO FUNCIONA EL FRONTEND

When a chrome window starts it always have the same file URL but with different
arguments. For this reason, it always loads the same HTML, containing all the
generic libraries. The arguments sent are: 

\begin{description}
	\item[view] Different depending on the window.
	\item[uuid] Window unique identifier. 
	\item[theme] Selects a different view palette.
\end{description}

When the application starts, there is no selected case so CaseSelector window 
shows up. This window just contains a folder picker to select the current case
to work.

Once selected, the window manager opens Case window. It is divided in tree
tools: Editor, Timeline and Search. Each of theme is in charge of a specific
analysis.

Editor tool gives a generic overview of the selected folder and files contained 
inside disk images. It has a file system watcher that notifies any modification
on the folder by dispatching action within Redux framework. Those are reduced
modifying the current state and build a file system tree inside a Javascript
object.

Sometimes actions are mapped to new ones using ReduxObservables. For instance,
when a raw disk image is detected inside the folder, a new action is dispatched
to notify the backend to perform a generic analysis to this image and to 
compute it's hash. This last one is used to guarantee proof integrity during
the analysis. The file system tree is extended with a deeper analysis that
retrieves files inside the image.

The Case file system tree has a context menu that allows the generation of
timelines. When generated, the Timeline tool has a table of all the actions 
performed to this disk image, for instance, file creations, modifications and
accesses.

Furthermore, the file system context menu has an item to perform search inside
images. Those searches will appear inside Search tool. It contains a table
with all the occurrences of the selected string. Finally there is another
option on the context menu to export files outside the image.


\chapter{Digital forensic tools}
\label{S:state-of-the-art}

There are many ways to extract information of digital sources that can be
useful to find clues to solve a case, and depending on it, investigator must use
a type of tool or another.

\begin{description}
	\item[Network tools] Network tools can capture the traffic generated from 
	the computer that is running the program.

	\item[Memory tools] They hare capable to acquire the current RAM memory of
	a computer and analyze it. They often have tools to list the processes
	running at that moment, get networking information (IP address, ARP tables,
	etc), among others.

	\item[Disk tools] They extract the whole information of a disk (or other
	physical memory storages) and analyze them without doing any modification
	to guaranty data integrity. They can list files inside the file system and
	recover the deleted ones. Sometimes they offer higher level functionalities 
	as generate timelines of the disk activity, get the history of internet
	navigation, etc.
\end{description}

Network tools must run at the exact same moment as the evidence is generated
but they can be executed on a different machine. By comparison, memory tools can
be executed after the proof generation but are lost when the machine turns off
and the acquisition must be executed directly on the data source. Finally, disk
tools clues can just be lost by removing the physical proof so they are easier 
to get. Also, to acquire the data, they only need the physical memory storage
and a specific hardware, called write-blocker.

There is a big amount of digital forensics tools but just a few that have a good
graphic user interface (GUI). And most of those user interfaces are developed 
windows users. Therefore, those two parameters must be considered.

Table \ref{T:digital-forensic-tools} contains some well-known tools used in
this subject. As we can see, regarding disk tools, Digital Forensics Framework 
(DFF), is an obsolete project with two years without any update. On the other
hand, we have Autopsy, that is a very good software but, even though it can be 
used on UNIX platforms, investigators don't use it because it is very difficult
to install. Furthermore, its technologies are quite older since newer cross
platform applications are developed using web technologies because they are now
very mature.

\begin{landscape}
\begin{table}
	\centering
	\begin{tabular}{|c||c|c|c|c|c|c|c|c|}
		% HEADER ---------------------------------------------------------------
		\hline 
		\uppercase{ & 
			\thead{Tools} & 
			\thead{open} & 
			\thead{GUI} & 
			\thead{win} & 
			\thead{OSX} & 
			\thead{Linux} & 
			\thead{comments}
		} \\
		\hline
		% BODY -----------------------------------------------------------------
		\makecell{ Digital Forensics \\ Framework }	& Disk &
		\checkmark & \checkmark & \checkmark & \checkmark & \checkmark &
		\makecell{
			At least two years without any update.
		} \\ \hline
		% ----------------------------------------------------------------------
		X-Ways Forensics & \makecell{ Disk \& \\ Application } &
		& \checkmark & \checkmark & & &
		{
			
		} \\ \hline
		% ----------------------------------------------------------------------
		EnCase & Multi-purpose &
		& \checkmark & \checkmark & & &
		{
			
		} \\ \hline
		% ----------------------------------------------------------------------
		Volatility & Memory &
		\checkmark & & \checkmark & \checkmark & \checkmark &
		{
		} \\ \hline
		% ----------------------------------------------------------------------
		Bulk Extractor & \makecell{ Memory \& \\ Disk } & 
		\checkmark & \checkmark & & \checkmark & \checkmark &
		{
		} \\ \hline
		% ----------------------------------------------------------------------
		The Sleuth Kit & Disk &
		\checkmark & & \checkmark & \checkmark & \checkmark &
		{
			
		} \\ \hline
		% ----------------------------------------------------------------------
		Autopsy & Disk &
		\checkmark & \checkmark & \checkmark & \checkmark & \checkmark &
		\makecell{
			From The Sleuth Kit creators. \\
			Difficult to compile in UNIX systems. \\
			Old technologies.
		} \\ \hline
		% ----------------------------------------------------------------------
		Wireshark & Network &
		\checkmark & \checkmark & \checkmark & \checkmark & \checkmark &
		{
		} \\ \hline
		% ----------------------------------------------------------------------
		Xplico & Network &
		\checkmark & \checkmark & & \checkmark & \checkmark &
		{
		} \\ \hline
		% ----------------------------------------------------------------------

	\end{tabular}
	\caption{Digital forensic tools}
	\label{T:digital-forensic-tools}
\end{table}
\end{landscape}


% Fuentes:
% https://resources.infosecinstitute.com/computer-forensics-tools/#gref

%% X-Ways Forensics

% https://www.youtube.com/watch?v=ggSXfAf4Eko&index=5&list=PLB0pU0wP67A9LezmyZO5I6DnHPEWjgjOD


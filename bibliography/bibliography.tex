%%% Per la bibliografia hi ha 2 opcions: generarla amb la utilitat BibTeX 
%%%                                      o fer-la ''a ma''
%%% NOTA: podeu trobar facilment informació sobre BibTeX a:
%%%  http://www.ctan.org/tex-archive/biblio/bibtex/contrib/doc/

%%% OPCIO 1: BibTeX (recomanat) -> descomentar les comandes seguents:
% \bibliographystyle{apacite}
\bibliographystyle{unsrt}   %% Estil de bibliografia EETAC 
\cleardoublepage \phantomsection % Indicar aqui el(s) fitxer(s) que contenen la bibliografia
\bibliography{plantilla_tfc}
% \pdfbookmark{<Bibliography}{sec:biblio}
%%% OPCIO 2: bibliografia manual
%%%
%%% L'argument d'entrada es el numero de referencies que s'inclouen
%\cleardoublepage \phantomsection \begin{thebibliography}{2}

%% Llibres:  Autor/s (cognoms i inicials dels noms), títol del llibre (en 
% cursiva), editor, ciutat i any de publicació. Quan es cita el capítol d'un
% llibre s'ha d'indicar el títol del capítol (entre cometes), el títol del
% llibre (en cursiva) i els números de pàgines amb la primera i la darrera
% incloses.

%%  Exemple de capitol en llibre
%\bibitem{prova1} Cognoms-autor, Inicial-nom.  ``Títol del capítol''. {\it Títol
%	del llibre}.  (Editor. Ciutat. Any publicació): pagina1--paginaN.

%%  Exemple de d'article en revista
%\bibitem{prova2} Cognoms-autor, Inicial-nom.  ``Títol de l'article''. {\it
%	Títol de la revista}.  {\bf volum}(numero), pagina1--paginaN. (Any
%		publicació) 

%\end{thebibliography}

